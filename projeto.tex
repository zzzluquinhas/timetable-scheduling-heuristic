% chktex-file 8
\documentclass[12pt]{article}

\usepackage{styles/sbc-template}
\usepackage{graphicx,url}
\usepackage[utf8]{inputenc}
\usepackage[brazil]{babel}

\graphicspath{{images/}}
     
\sloppy

\title{Projeto de pesquisa}

\author{
	Chico\inst{1},
	Juan Marcos Braga Faria\inst{1},
	Lucas Almeida Santos de Souza\inst{1},
	Pep\inst{1}
}

\address{
	Departamento de Ciência da Computação\\
	Universidade Federal de Minas Gerais (UFMG) -- Belo Horizonte, MG -- Brasil
	\email{\{chico,juanmarcos,lucasalmeida,pep\}@dcc.ufmg.br}
}

\begin{document} 

\maketitle

\begin{abstract}
	Abstract vai aqui~\cite{cormen:2009} % aqui só pra não dar erro, remover depois
	% \textbf{CAPRICHAR NA INTRODUÇÃO!!}. Outra coisa que pode ajudar: definir claramente 1-dados do problema, 2-variaveis de decisão (subconjunto dos dados escolhido), 3-restrições do problema e 4-métrica que indica que uma solução é melhor que outra. É quase como se fosse um trabalho de PO
\end{abstract}
     
\begin{resumo}
	O objetivo deste trabalho é desenvolver a habilidade de escrever um projeto de pesquisa científica na área de heurísticas para problemas de otimização. O trabalho será avaliado pela sua clareza e completude, ou seja, ele deve descrever precisamente qual é o problema que será abordado, (ii) quais serão exatamente as hipóteses investigadas, e (iii) quais os experimentos que serão realizados para verificar estas hipóteses. 

	Um exemplo de uma pesquisa muito comum nesta área é a proposta de uma nova heurística para um problema NP-Difícil. Sendo assim, a hipótese investigada é se esta heurística obtém resultados competitivos com as melhores heurísticas já descobertas para este problema. Recomenda-se que o aluno faça uma pesquisa com este formato, mas ele está livre para propor uma pesquisa em outro formato. O aluno pode trabalhar com qualquer problema NP-Difícil, exceto aquele abordado no trabalho de implementação. 

	Para facilitar a escrita e a correção deste projeto, ele deverá ter exatamente a seguinte estrutura de seções, onde cada subtópico é composto de preferencialmente (mas não obrigatoriamente) um parágrafo. 
\end{resumo}

\section{Introdução}
%(apresento a proposta)
\subsection{Contextualização}
%(do problema real)
\subsection{Definição do problema}
% (de otimização)
\subsection{Caracterização do problema}
% (Classe de complexidade)
\subsection{Caracterização do problema}
% (Problemas semelhantes, caso hajam)
\subsection{Caracterização do problema}
% (Qualquer outra informação relevante)
\subsection{Motivação}
% (Por que vale a pena estudar este problema)
\subsection{Objetivos}
% (qual metaheurística que você pretende aplicar na solução)
\subsection{Justificativa}
% (Por que é importante atingir esses objetivos)
\subsection{Roteiro}
% (O que esperar nas próximas seções)

\section{Trabalhos Relacionados}
% (situa o leitor no estado-da-arte na solução do problema) 
% Exatamente um parágrafo resumindo cada trabalho relacionado, contendo:
% - Tópico frasal descrevendo qual é a relação com o tema do projeto
% - Resumo dos resultados do trabalho 
% - Resumo das conclusões do trabalho

\section{Metodologia}
% (que será empregada para solucionar o problema) 
\subsection{Detalhes sobre a heurística de controle}
% (incluindo pseudocódigo) 
\subsection{Detalhes sobre a heurística proposta}
% (incluindo pseudocódigo)
\subsection{Planejamento dos experimentos}
% (incluindo os de ajuste de parâmetros)
%Para cada uma das subsubsections abaixo, justificar quais as observações que você espera fazer.
\subsubsection{Detalhes das instâncias de teste que serão utilizadas}
\subsubsection{Detalhes das tabelas e gráficos que serão apresentados}

\section{Cronograma}
\subsection{Etapas}
% (dividir e numerar cada etapa necessária para executar a metodologia)
\subsection{Prazos}
% (atribuir uma parcela do tempo total de execução para cada etapa)

\section{Equipe e Atribuições}
% Breve descrição sobre cada membro da equipe e suas atribuições dentro do projeto
% Todos os membros devem participar do design, implementação e descrição, de alguma das heurísticas que compõem o projeto. 


\section{Considerações Finais}
% No caso de que haja algo a mais a ser dito


\bibliographystyle{styles/sbc}
\bibliography{projeto}
% bibliografias:
% Todas as que são citadas no projeto.
% Completas e padronizadas (no formato da SBC)

\end{document}
